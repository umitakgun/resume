\documentclass[12pt,letterpaper]{article}

\def\myauthor{Ibrahim Umit Akgun}
\def\mytitle{umit-cv}
\def\myemail{i.umitakgun@gmail.com}
\def\myedumail{ibrahim.akgun@ozu.edu.tr}
\def\myweb{umitakgun}
\def\linkedin{https://tr.linkedin.com/in/ibrahim-ümit-akgün-29a0a942}
\def\myphone{+90 5078740393}
\def\mykeywords{
  umit akgun, 
  resume, 
  curriculum, 
  vita, 
  curriculum vita, 
  cv, 
  umit, 
  akgun, 
}

\setlength\parindent{0pt}

\newenvironment{itemize*}%
{\begin{itemize}%
  \setlength{\itemsep}{0pt}}%
{\end{itemize}}

\usepackage{url}
\usepackage{fancyhdr}
\usepackage{lastpage}
%\usepackage{ocgtools}
\usepackage{enumitem}
\setlist{nolistsep,leftmargin=0.15in}
\usepackage[log-declarations=false]{xparse}
\usepackage[adobe-garamond]{mathdesign}
\usepackage[no-math]{fontspec}
\usepackage{microtype}
\usepackage[margin=1.125in,top=1.375in,right=1in,left=2in]{geometry}
\usepackage[strict]{changepage}
\usepackage[
  ocgcolorlinks,
  urlcolor={[rgb]{0,0,0.54}},
  unicode,
  plainpages=false,
  pdfpagelabels,
  pdftitle={\mytitle},
  pdfauthor={\myauthor},
  pdfkeywords={\mykeywords}
]{hyperref}

% fix ocgcolor link breaking; thanks due to Benjamin Lerner (http://goo.gl/VZKR7M)
\makeatletter
\AtBeginDocument{%
  \newlength{\temp@x}%
  \newlength{\temp@y}%
  \newlength{\temp@w}%
  \newlength{\temp@h}%
  \def\my@coords#1#2#3#4{%
    \setlength{\temp@x}{#1}%
    \setlength{\temp@y}{#2}%
    \setlength{\temp@w}{#3}%
    \setlength{\temp@h}{#4}%
    \adjustlengths{}%
    \my@pdfliteral{\strip@pt\temp@x\space\strip@pt\temp@y\space\strip@pt\temp@w\space\strip@pt\temp@h\space re}}%
  \ifpdf
    \typeout{In PDF mode}%
    \def\my@pdfliteral#1{\pdfliteral page{#1}}% I don't know why % this command...
    \def\adjustlengths{}%
  \fi
  \ifxetex
    \def\my@pdfliteral #1{\special{pdf: literal direct #1}}% isn't equivalent to this one
    \def\adjustlengths{\setlength{\temp@h}{-\temp@h}\addtolength{\temp@y}{1in}\addtolength{\temp@x}{-1in}}%
  \fi%
  \def\Hy@colorlink#1{%
    \begingroup
      \ifHy@ocgcolorlinks
        \def\Hy@ocgcolor{#1}%
        \my@pdfliteral{q}%
        \my@pdfliteral{7 Tr}% Set text mode to clipping-only
      \else
        \HyColor@UseColor#1%
      \fi
  }%
  \def\Hy@endcolorlink{%
    \ifHy@ocgcolorlinks%
      \my@pdfliteral{/OC/OCPrint BDC}%
      \my@coords{0pt}{0pt}{\pdfpagewidth}{\pdfpageheight}%
      \my@pdfliteral{F}% Fill clipping path (the url's text) with current color
      \my@pdfliteral{EMC/OC/OCView BDC}%
      \begingroup%
        \expandafter\HyColor@UseColor\Hy@ocgcolor%
        \my@coords{0pt}{0pt}{\pdfpagewidth}{\pdfpageheight}%
        \my@pdfliteral{F}% Fill clipping path (the url's text) with \Hy@ocgcolor
      \endgroup%
      \my@pdfliteral{EMC}%
      \my@pdfliteral{0 Tr}% Reset text to normal mode
      \my@pdfliteral{Q}%
    \fi
    \endgroup
  }%
}
\makeatother
% end fixes

\newcommand{\Csh}{C{\#}}

\newcommand{\ML}{\textsc{Matlab}}
\newcommand{\Simu}{Simulink}
\newcommand{\MLS}{\ML{}/\Simu{}}
\newcommand{\apdl}{\textsc{APDL}}
\newcommand{\ansys}{\textsc{Ansys}}
\newcommand{\fluent}{\textsc{Fluent}}
\newcommand\CPP{C/C\ensuremath{+}\ensuremath{+}}
\newcommand{\Star}{\textsc{Star-CCM\ensuremath{+}}}

\newcommand{\mhead}[1]{\leavevmode\marginpar{\sffamily\footnotesize #1}}
\newcommand{\rdate}[1]{{\addfontfeature{Numbers=OldStyle} \hfill #1}}
\renewcommand{\date}[1]{{\addfontfeature{Numbers=OldStyle} #1}}
\renewcommand{\labelitemi}{-} 

\setmainfont[
  Ligatures={TeX,Common},
  BoldFont={AGaramondPro-Semibold},
]{Adobe Garamond Pro}
\setsansfont[
  Ligatures={TeX,Common},
  Letters=SmallCaps,
  Color=660000,
]{Adobe Garamond Pro}
\setmonofont[Scale=0.85]{FontAwesome}

\makeatletter % fix for \hrulefill w/ mathdesign package
\def\hrulefill{\leavevmode\leaders \hrule height \rulethickness \hfill\kern\z@}
\makeatletter

\begin{document}\flushbottom
\pagestyle{fancy} \setlength\headwidth{6.5in}
\rhead{\textsc{I. Umit.~Akgun—r\'{e}sum\'{e}—\thepage{} of \pageref*{LastPage}}} \cfoot{}
\thispagestyle{empty}
\begin{adjustwidth}{-1in}{}
{\Huge
  {\textsc{%
    {\addfontfeature{Style=TitlingCaps}I}\kern-2pt.~% 
    {\addfontfeature{Style=TitlingCaps}U}\kern-1.5ptmit
    {\addfontfeature{Style=TitlingCaps}A}kgun}
  }
}
\hfill\hfill\hfill
{
  \hfill
  \begin{minipage}[b]{1.5in}
    \flushright \footnotesize 
    \href{tel:\myphone}{\myphone} \\ %\texttt{}~
	No military obligation \\    
    \href{mailto:\myemail}{\myemail} \\
    \href{mailto:\myedumail}{\myedumail} \\
    \href{https://www.github.com/\myweb}{\texttt{}}~\href{\linkedin}{\texttt{}~\myweb}
  \end{minipage}
}\par
\hrulefill
\end{adjustwidth}  
\reversemarginpar 
\setlength\marginparwidth{0.85in}
\smallskip
\mhead{Education}%
\textbf{Ozyegin University,} Istanbul, Turkey \newline
\emph{Master of Science, \href{http://www.ozyegin.edu.tr/AKADEMIK-PROGRAMLAR/Muhendislik-Fakultesi/Computer-Science?lang=en-US}{Computer Science}} \rdate{2011–2014}
\begin{itemize*}
  \item Cumulative GPA: 3.33/4.00; {Compiler Optimization}
  \item Thesis: \href{https://dl.dropboxusercontent.com/u/48518192/umitakgun_thesis.pdf}{Performance Evaluation of Unfolded Sparse Matrix-Vector Multiplication}
  \item Advisor: Asst. Prof.~T. Bar{\i}\c{s}.~Aktemur
  \item Research Interests: Compiler design, programming language design and
semantics, runtime program generation, software analysis, software engineering
\end{itemize*}

\medskip
\textbf{Ege University,} Izmir, Turkey \newline
\emph{Bachelor of Science, \href{http://bilmuh.ege.edu.tr/}{Computer Engineering}} \rdate{2005-2009}
\begin{itemize*}
  \item Cumulative GPA: 3.18/4.00; {Operating Systems, Embedded Systems}
  \item Senior Project:{Operating System For Wireless Sensor Networks : SIMIT}
  \item Advisors: Prof.~Aylin~Kantarc{\i}
  \item Research Interests: Operating Systems, Wireless Sensor Network, Network
Programming, Embedded Systems.
\end{itemize*}

\medskip
\textbf{MEV Izmir Science High School,} Izmir, Turkey 
\rdate{2005}
\begin{itemize*}
  \item GPA: 4.96/5.00
\end{itemize*}

\bigskip
\mhead{Research \newline Interests}%
Compiler Design, Operating Systems, Programming Language Design and Semantics,
Multicore Programming, Runtime Program Generation, Software Analysis, Software
Engineering, Domain Specific Languages, Embedded and Real-Time Systems, Network
Programming, Distributed and Parallel Computing.

\bigskip
\mhead{Software \newline Proficiencies}%
\begin{itemize}
  \item \emph{Fluent} \CPP{},Java, Swift, Objective-C, Scala
  \item \emph{Working Knowledge} C{\#}, Python, Haskell, Rust, Perl
  \item \emph{Scripting, Typography} Bash, Ksh, Zsh, Lex Yacc, \LaTeX{}
  \item \emph{Assembly} i386, x86/64(SIMD), ARM, PowerPC   
  \item \emph{Development Environment} XCode, Android Studio, Emacs, Vim, Git, IntelliJ, Eclipse, SVN, Linux, OS X, Windows, 
\end{itemize}

\bigskip
\mhead{Professional \newline Experience}%
\textbf{ING Bank,} Istanbul \newline
\emph{Software Engineer (iOS Development)} \rdate{\textsc{Aug}~2014–\textsc{...}} \newline
I was working on ParaMara application which is designed and developed by Objective-C. I was responsible for coding key features of ParaMara. Currently, I'm working on New ING Mobile application which is designed and developed by pure Swift. I'm leading iOS Team which consists of four developers and mostly working on creating a new flexible and stable security framework for financial mobile applications.  

\textbf{SIEMENS,} Istanbul \newline
\emph{Software Engineer (Embedded Systems and Real-time Frameworks)} \rdate{\textsc{May}~2013–\textsc{Aug}~2014} \newline
I was working on real-time and embedded systems especially Software PLC. We developed with C++ applying some best practices and design patterns. We built a framework for Siemens PLCs which are widely used in automation systems. My specialties based on software design and architecture. I was working
on network communication part of PLCs as well. We developed multi-threaded network
oriented programs.


\medskip
\textbf{TUBITAK (The Scientific and Technological Research Council
of Turkey equivalent NSF),} Istanbul \newline
\emph{Software Engineer (Embedded Systems and Operating System)} \rdate{2010–2013}\newline
I was working on real-time and embedded operating systems. Extensive C
programming skills on embedded real-time systems for avionics software. I have
involved multithreaded applications, memory management, pthread library, thread
manager and memory manager for our operating system. I gained solid debugging skills to
solve system problems, maintenance of lightweightIP networking stack written by
Adam Duncan; I have a desire for high quality software systems, drives juniors to excel
in software engineering principles. Our team were consisting of five software engineers. Also I
was working on multicore concurrent programming. I implemented lock-free data
structures.(linked list, priority queue).

\medskip
\textbf{Ozyegin University,} Istanbul \newline
\emph{Research Assistant} \rdate{2011–2014}\newline
I was doing research on LLVM, code generation, compiler design and programming
languages at University of Ozyegin. (You can reach more information in Runtime
Program Generation and Empirical Optimization section )

\medskip
\textbf{IBM Global Services,} Istanbul \newline
\emph{UNIX/Linux System Administrator} \rdate{2009–2010}
I was UNIX/Linux administration for IBM Global Services. I have worked on
management of SAP and DB2. In addition, I also worked on database backup
recovery operations.

\medskip
\textbf{Ege University Computer Engineering Department,} Izmir \newline
I wrote interprocess communication systems and kernel memory management
algorithms for eGIS operating that is written by Kasim Sinan Yildirim. And I had
many contribution to several part of (real-time concept) eGIS. During the internship,
researched real-time operating systems and developed several embedded applications
such as QNX operating system and Wireless Sensor Network Routing Algortihms.


\bigskip
\mhead{Honors \& \newline Awards}%
Ozyegin University, Full Tuition Scholarship         \rdate{2011-2014}\newline
MEV Izmir Science High School Physic Olympiad School Team	\rdate{2003-2005}

\newgeometry{margin=1.125in,top=1in,right=1in,left=2in}
\reversemarginpar\setlength\marginparwidth{0.85in}
\mhead{Thesis}%
\textbf{MSc. : Performance Evolution Of Unfolded Matrix-Vector Multiplication Code} \newline
Sparse matrix-vector multiplication (spMV) is a kernel operation in scientific
computation. There exist problems where a matrix is repeatedly multiplied by many
different vectors. For such problems, specializing the spMV code based on the matrix
has the potential of producing significantly faster code. This, in fact, has been one of
the motivational examples of program generation. Using program generation, spMV
code can be unfolded fully to eliminate loop overheads as well as enable optimizations
with high impact. In this dissertation we focus on specialization of spMV by unfold-
ing the code according to a given matrix. We provide an experimental evaluation of
performance using 70 sparse matrices collected from real-world scientific computation
domains. We present optimizations with which high-performant assembly code can be
generated much more rapidly than the general-purpose compiler icc. We finally
present how one of the optimizations we studied can be integrated into a compiler as a
code transforming pass.
\newline\newline
\textbf{BSc. : Embedded Operating System For Wireless Sensor Networks}\newline
My thesis is on an operating system. WSN’s operating systems are embedded
operating systems(Some of them RTOS). So the efficiency and robustness were kept
in mind when developing the operating system. I examined other WSN’s operating
systems and several realtime operating systems. I worked on thread management, IPC
and scheduling algorithm especially and trying to progress in these subjects. The
operating system was developed completely with the help of open-source tool ( gcc,
gdb, ddd ...). ( whole development process on GNU/Linux (Debian)).

\bigskip
\mhead{Recently worked on\newline Projects}%
\textbf{Runtime Program Generation and Empirical Optimization}\newline
This project is generously funded by TUBITAK.
The goal of this project is to combine the advantages of runtime program generation
(RTPG) with empirical optimization (EO). RTPG can specialize a program according
to runtime inputs that are not available at compile-time, hence provides data-specific
optimization opportunities. Empirical Optimization (aka auto-tuning) aims to find the
most efficient version of a program on a specific architecture by performing install-time
experiments. The project will explore the combination of these two techniques; in
particular, how we can use empirical optimization to drive the decisions taken during
runtime generation.
\newline
\href{http://srl.ozyegin.edu.tr/projects/rtpg4llvm}{Compiler for runtime program generation}\newline
\href{http://srl.ozyegin.edu.tr/projects/spMVlib/}{Sparse matrix-dense vector multiplication library}

\medskip
\textbf{Software Pattern Recognition}\newline
This project main purpose was software pattern search in source code. These patterns
are well know design patterns or user defined patterns. Later developments took place
within the scope of the project. I extracted AST(Abstract Syntax Tree) from the
source code and searching patterns in this AST. Our ultimate goal was not only own
patterns, also we recognize user defined relationships. This project has been written in
Java and it targets only Java code as an input. I wrote many software design pattern
code. And I trained my program these samples. My program extract all AST’s of these
design patterns. We tried well-known Java frameworks to our program. As you know,
some design patterns have a similar software model. Our program search design
pattern AST’s in given code and generate statistical data. These data shows us which
part of program similar to which design pattern.

\bigskip
\mhead{Languages}
\begin{itemize*}
  \item English : Fluent (TOEFL score : 96)
  \item Turkish : Native
\end{itemize*}

\bigskip
\mhead{References}%
\begin{itemize*}
\item \href{http://aktemur.github.io}{Dr. Tankut Baris Aktemur, Assistant Professor}
\item \href{http://faculty.ozyegin.edu.tr/ismailari}{Dr. Ismail Ari, Assistant Professor}
\item \href{http://ugurdag.com}{Dr. Fatih Ugurdag, Associate Professor}
\item \href{http://yzgrafik.ege.edu.tr/~ugur}{Dr. Aybars Ugur, Associate Professor}
\item \href{http://efe.ege.edu.tr/~kantarci}{Dr. Aylin Kantarci, Associate Professor}
\end{itemize*}

\end{document}
